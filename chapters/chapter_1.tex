\chapter{Introduction}
\section{DNA damage responses in cells}
Living organisms are continuously subjected to external and internal conditions that cause damage to their genetic material. It is estimated that a single mammalian cell is susceptible to upto 100,000 damaging events in a day \cite{ciccia2010dna}. When left unrepaired, such damages can result in mutations that can give rise to the downstream development of different cancers, and neurodegenerative diseases \cite{friedberg2005dna}. To overcome this problem of damage and error propagation in genetic information, cells have evolved diverse error detection and correction mechanisms, collectively known as DNA damage responses (DDR) \cite{hoeijmakers2009dna}. DDR often utilizes post-translational modifications (PTM), such as ubiqutination, phosphorylation, methylation etc. to propagate damage signals to downstream transducer proteins, like ATM or ATR. These transducers further effect higher-level decisions, such as checkpoint arrest to give the cell more time to repair, or if the damage is too extensive, proceed to apoptosis \cite{derks2014dna}.

\subsection{Features of DNA}
\paragraph*{} All known life on Earth stores their hereditary information in deoxyribose nucleic acid (DNA), a molecule that can be chacterised as a continous, unbranched linear chain, which is paired to a complementary strand to make the double helix. DNA is made of a deoxyribose sugar with a phosphate group between sugars, and to each sugar is attached a nitrogenous base. The combination of the sugar, phosphate group and the nitrogenous base is known as the nucleotide. There are four nitrogenous bases commonly found in DNA, which are Adenine (A), Guanine (G), Thymine (T) or Cytosine (C). The bases in opposite strands of DNA are complimentary in nature, with A pairing opposite T, and G pairing opposite C which allows for data redundancy at the encoding level on the storage medium.

\paragraph*{} Much of the chemically encoded information of the genetic material is thought to be in the storage of the synthesis sequence of a polypeptide chain by ribosomes. In the coding sequence of the DNA, a unit of three nucleotide (called codons) is known to be encoding an amino acid. This is established as the genetic code, with a combination of 64 three letter nucleotide sequence coding for 20 different amino acids commonly found in the proteins. There are start and stop codons that define the boundary of coding sequence in the linear chain of DNA. Since there are more codons than amino acid, more than a single codon can encode for an amino acid. For instance, for Lysine, the three-letter codes are AAA and AAG. Further discussion on this can be found on many standard molecular biology texts.

\subsection{DNA damage responses}
\paragraph*{} DNA, because of its chemical nature, has a tendency to react with the chemical environment that surrounds it. This makes the information stored in this molecule to become inherently unstable over time, as the chemical environment can gradually change the molecule in different ways. Largely, chemical reactions can affect the genetic material in the level of the nitrogenous base, the nucleotide, and the phosphodiester bond. The base can be subjected to damages due to spontaneous processes such as deamination, or methylation. There can also be oxidative damage to the amino groups due the prescence of free radicals.


\paragraph*{} For convinience, the sources of damage can be largely grouped as factors outside the cell, or exogenous sources, and factors inside the cell, arising from the overall cellular machinery, which are endogenous to the cells. Exogenous sources may include ionizing radiations from space, high energy photons from Sun, or carcinogenic industrial chemicals that are present in the environment due to human activity. Endogenously, chemical biproducts in biogenesis results in the formation of free radicals, which when in proximity to another molecule can freely exchange electrons, changing the electronic configurations of molecules, creating unstability.

\paragraph*{} DNA that is damaged has to be repaired. Since there are several types of damage that are chemically distinct, it is not surprising that natural selection has evolved a series of mechanisms to sense and repair damage. Also, unlike other DNA processes, such as replication or transcription, damages to DNA are stochastic, which requires repair systems to have a close watch on the molecular health of the DNA. The repair mechanism is an event based system that triggered by a chemical damage event. This further triggers downstream recruitment of repair factors to the site of damage.

\subsection{Repair in eukaryotes}
\paragraph*{} In eukaryotes, DNA is packaged as chromatin with the help of nucleosomes. In-vitro experiments have shown that the presence of nucleosomes hampers repair processes such as base excision repair (BER) and nucleotide excision repair (NER) \cite{hara2000dna, odell2011nucleosome}. This further increases the complexity of our understanding of DDR, as the repair processes have to seamlessly operate within the context of chromatin, which is a refractory environment for repair. However, chromatin modifying factors, such as SWI/SNF, histone acetyletransferases (HATs) have been shown to be recruited to sites of damage \cite{park2006mammalian, polo2010regulation}. Chromatin integrity factor, KAP-1, has been found to be a substrate of ATM, which shows chromatin modulation can be a downstream transducer effect of DDR \cite{ziv2006chromatin}. These results indicate that there exists an interplay between chromatin and DDR, which have traditionally been seen as distinct fields of study.

\paragraph*{} It has been shown that damage repair characteristics vary across the nucleus depending on the density of chromatin packaging. Markers of damage are seen to persist for longer in sites of densely packed DNA (heterochromatin regions) as compared to sites of loosely packed DNA (euchromatin regions) \cite{goodarzi2008atm}.

\paragraph*{} Since DDR processes and their complex boundaries with other known and unknown biological systems are not well-understood, it presents an opportunity to develop tools and techniques from across the fields of microscopy, robotics and image processing to study such biological processes in scale. While there have been studies of DDR in living cells using tools of live cell microscopy, these have used indirect readouts of chromatin compaction states and also have been limited in terms of throughput. Progress towards uncovering DDR would immensely help in addressing fundamental problems related to biological circuitry, such as stochasticity in responses, and further our knowledge of disease biology.

\subsection{Methods to study DDR in chromatin}
\paragraph*{} To this end in this thesis, I first describe the development of fluorescence anisotropy imaging tools to investigate chromatin compaction states upon laser-induced double strand breaks. Further to improve the throughput of such microscopic investigations we develop methods that improve statistics in both fixed and live cell investigations of DDR.



DNA in the eukaryotic cell nucleus is packaged into chromatin with histones and other protein components. This genetic material is susceptible to damage from different sources. It is estimated that in a day, a single mammalian cell can face as many as 100,000 lesions to its DNA \cite{ciccia2010dna}. If left unrepaired, such damage can result in cell cycle arrest, cell death, or senescence, or, at the level of the organism, cause mutations that result in diseases such as cancers or neurodegenerative diseases \cite{friedberg2005dna, madabhushi2014dna}. To deal with this constant assault on genetic material, cells have evolved a cohort of mechanisms that sense and repair DNA damage \cite{hoeijmakers2009dna}.

Every DNA damage response (DDR) in eukaryotic cells takes place in the context of chromatin. Biochemical studies in in vitro systems with synthetically damaged DNA have helped uncover the critical role that chromatinization plays in DDR. In experiments with purified systems of repair, when a nucleosome was added to the naked DNA, measurements of repair kinetics by nucleotide excision repair (NER) and base excision repair (BER) indicated that the repair was much slower than that of naked DNA without a nucleosome \cite{hara2000dna,odell2011nucleosome}. This led to the conclusion that the proteins that help package DNA inside the nucleus can have a hindering effect on damage repair. In cells, regions undergoing repair have been found to be more sensitive to micrococcal nuclease digestion than bulk DNA \cite{smerdon1978distribution}. This indicates that DNA can be more exposed during damage repair. Proteins that modify nucleosomes, such as SWI/SNF, and histone acetyltransferases (HATs), such as CHD4, have been found to be recruited to the site of damage, which indicates that the chromatin at the site of damage may be relaxed as a response to damage \cite{park2006mammalian, polo2010regulation}. Further proteins that maintain chromatin integrity, such as KAP-1, have been found to be substrates for the DDR master kinase, ATM \cite{ziv2006chromatin}.

A previous study has found that markers of repair persist in heterochromatin regions for longer time than in euchromatic regions \cite{goodarzi2008atm}, indicating that the dynamics of repair is sensitive to the compaction state and activity of chromatin. With core histone H2B tagged with GFP and microirradiation of Hoechst-sensitized cells, it has been shown that chromatin is decompacted at the site of damage in response to clustered double-strand breaks (DSBs), followed by a phase of increased compaction \cite{kruhlak2006changes,strickfaden2016poly}. Compaction of chromatin is critical to DDR, and compaction in the absence of damage is sufficient to stimulate a damage response, independent of damage \cite{BURGESS20141703}. However, in experiments using microscopy, the spreading or shrinking of a chromatin region marked with a specific fluorescent histone has been used as a proxy for chromatin compaction, and as such does not report directly on physical state of the chromatin. In other studies, fluorescence anisotropy imaging (FAI) has been used to map the compaction of chromatin in living cells directly \cite{banerjee2006chromatin}. Core histone H2B tagged with EGFP is excited with polarized light, which preferentially excites EGFP molecules whose excitation dipoles are oriented along the polarization axis of the excitation light. The extent of depolarization of the emission light gives a measure of the rotational diffusion of EGFP fusion proteins. The higher the rotational diffusion, the greater is the extent of depolarization of the emission signal, over and above what would be expected because of random orientations of the fluorophores. Fluorescence anisotropy is a measure of the extent of depolarization \cite{lakowicz2013principles,ghosh2012dynamic}. Since H2B-EGFP in the regions of euchromatin should have greater rotational mobility than regions of heterochromatin, anisotropy maps generated by FAI show evidence of differential compaction of chromatin and as such could be used as a direct physical measure of local chromatin packaging \cite{bhattacharya2009spatio}. In this study, we sought to use FAI in the context of DDR to monitor chromatin compaction states directly. Chromatin decompaction is essential for repair, and yet local chromatin compaction may be used by cells to prevent further damage \cite{BURGESS20141703}. Using FAI, we studied the physical changes to the chromatin structure in response to laser microirradiation–induced clustered DSB, in regions of chromatin beyond just the site of damage. We show that anisotropy maps are preserved in fixation and regions of high and low anisotropy indeed correspond to physiologically relevant markers for heterochromatin and euchromatin, respectively. This also allows us to first follow compaction changes in response to localized DSBs in living cells, and then fix the cells and perform immunofluorescence for markers of DNA damage. Finally, we follow the differential dynamics of two endogenous damage-responsive proteins (PCNA and PARP1) with respect to chromatin compaction maps and show that their time scales of recruitment and subsequent dispersion are very different. In addition to being recruited at the site of damage, PCNA also forms nodes further away in regions of low anisotropy. These PCNA nodes in open chromatin incorporate deoxynucleotide analogs, indicating that individual DSBs from the laser-induced cluster may be extruded out from the site of damage for the purposes of repair. Together, these studies open up a new avenue of following DDR in live cells in the chromatin, while also taking advantage of different immunofluorescent markers for DNA damage and chromatin.