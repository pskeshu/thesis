\chapter{Conclusion}
In conclusion, we have enhanced the existing methods to investigate DNA damage responses in live and fixed cells. Using our tools, we have been able to damage DNA with microirradiation and observe chromatin compaction states over time. We have observed that even undamaged chromatin is compacted upon damage, and repair factors are recruited to sites of less compact chromatin, where repair processes are detected. We observed the formation of these nodes of repair, which incorporate EdU and lie in open regions of chromatin by following PCNA induction in live cells post microirradiation. These experiments led to the realization that microscopy experiments fundamentally suffer from a  problem of throughput: the lack of an advanced software, which limits their potential. Towards overcoming this challenge, we built a prototype software to run automated experiments in large scale, generating high throughput data, with higher spatio-temporal resolution. This has allowed us to capture rare events in live cells, without the need to synchronize cells with cell cycle blockers. In the days to come such tools will be critical for microscopic investigations of DDR.
